\documentclass[12pt, letterpaper]{article}
\usepackage[utf8]{inputenc}
\usepackage{amssymb}

\title{Task 2}
\author{Vladislav Hober}
\date{}
\begin{document}

\maketitle

\section{Triangle inequality}
Let's assume that 
\begin{math}dist(x, z) > dist(x, y) + dist(y, z)\end{math}.
So, dist(x, z) is the shortest path between x and z. However, we can find a shorter path between these vertices:
\begin{math}dist(x, z) = dist(x, y) + dist(y, z)\end{math}.
It's a contradiction. Consequently, for any connected graph
\begin{math}dist(x, z) \leq dist(x, y) + dist(y, z)\end{math}.

\section{Tree}
G is a connected graph. Let's consider 2 cases.
\begin{itemize}
  \item \begin{math} |E| < |V| - 1 \end{math}.\newline
  Then some vertices will not have edges. So, the graph will not be connected. It is a contradiction.
  \item \begin{math} |E| > |V| - 1 \end{math}.\newline G is a connected graph. So, \begin{math}|V|-1\end{math} edges should connect all vertices. There is no cycles in a graph because with \begin{math}|V|-1\end{math}  we can create only a one path between all \begin{math}|V|\end{math} vertices. So, if we add more than \begin{math}|V|-1\end{math} edges, we will create a new path. Two different paths between vertices would create a cycle. We would get a cyclic graph.
\end{itemize}
 Consequently, a connected graph with only \begin{math}|V|-1\end{math} edges would be acyclic, would be a tree.
\section{Whitney}
Let be \begin{math}\varkappa(G) = n\end{math},
\begin{math}\lambda(G) = m \end{math},
\begin{math}\delta(G) = q \end{math}.
\begin{itemize}
  \item Let's assume that \begin{math} n > q \end{math}.\newline But if some vertex has q as a minimum degree, we can just remove q vertices to disconnect graph. So, \begin{math}\varkappa(G)\end{math} equals at least q. We came to contradiction. Consequently, \begin{math} n \leq q \end{math}.
  \item Let's assume that \begin{math} m > q \end{math}.\newline But if some vertex has at least q edges, we can just remove q edges to disconnect graph. So, \begin{math}\lambda(G)\end{math} equals at least q. We came to contradiction. Consequently, \begin{math} m \leq q \end{math}
\end{itemize}
In conclusion, \begin{math} \varkappa(G) \leq \lambda(G) \leq \delta(G)\end{math}

\section{Chartrand}
Let's prove by induction. It's a correct statement for graphs that consist 1, 2 or 3 vertices. Assume that this statement is correct for a graph with n vertices. Let's prove that this statement also is correct for a graph with (n+1) vertices.\newline A new vertex of a graph must have \begin{math}|V/2|\end{math} or more edges. Besides, only removing all new edges can disconnect a graph because it has already been connected by previous edges. So, if we add a new vertex, we won't be able to get better answer for \begin{math}\lambda(G)\end{math}.Consequently,  \begin{math}\delta(G) = \lambda(G) \end{math} for a graph with (n+1) vertices.
\section{Menger}
Let's assume that the size of the minimum vertex cut does not equal to the maximum number of pairwise internally vertex-disjoint paths between some vertices v and u. Consider 2 cases.
\begin{itemize}
  \item The size of the minimum vertex cut is less than the maximum number of paths.\newline
  So, the number of vertices we must delete to disconnect a graph is less than the number of paths. But paths are pairwise internally vertex-disjoint. Consequently, to "destroy" all paths between v and u we must remove at least a vertex from every path. We came to contradiction. The size of vertex cut couldn't be less than the maximum number of paths.
  \item The size of the minimum vertex cut is larger than the maximum number of paths.\newline
  Removing one vertex from every pairwise internally vertex-disjoint paths is enough to "destroy" all paths between v and u and disconnect a graph. So, vertex cut is not minimum. We came to contradiction.  
\end{itemize}
In conclusion, the size of the minimum vertex cut equal to the maximum number of pairwise internally vertex-disjoint paths between some vertices v and u.

\section{Harary}
A block of a block graph always has two vertex-disjoint paths between vertices (definition of 2-vertex-connectivity). Assume that some vertices of a block are not connected. But it couldn't be true because they have a path between them. So, they have same vertex cut. They should be connected. \newline
Consequently, all of vertices of a block of a block graph are connected. It is a clique.


\end{document}
