\documentclass[12pt, letterpaper]{article}
\usepackage[utf8]{inputenc}
\usepackage{amssymb}

\title{Graph theory homework, task 2}
\author{Vladislav Hober}
\date{}
\begin{document}

\maketitle

\section{Triangle inequality}
Let's assume that 
\begin{math}dist(x, z) > dist(x, y) + dist(y, z)\end{math}.
So, dist(x, z) is the shortest path between x and z. However, we can find a shorter path between these vertices:
\begin{math}dist(x, z) = dist(x, y) + dist(y, z)\end{math}.
It's a contradiction. Consequently, for any connected graph
\begin{math}dist(x, z) \leq dist(x, y) + dist(y, z)\end{math}.

\section{Tree}
G is a connected graph. Let's consider 2 cases.
\begin{itemize}
  \item \begin{math} |E| < |V| - 1 \end{math}.
  Then some vertices will not have edges. So, the graph will not be connected. It is a contradiction.
  \item \begin{math} |E| > |V| - 1 \end{math}. G is a connected graph. So, \begin{math}|V|-1\end{math} edges should connect all vertices. There is no cycles in a graph because with \begin{math}|V|-1\end{math}  we can create only a one path between all \begin{math}|V|\end{math} vertices. So, if we add more than \begin{math}|V|-1\end{math} edges, we will create a new path. Two different paths between vertices would create a cycle. We would get a cyclic graph.
\end{itemize}
 Consequently, a connected graph with only \begin{math}|V|-1\end{math} edges would be acyclic, would be a tree.
\section{Whitney}
Let be \begin{math}\varkappa(G) = n\end{math},
\begin{math}\lambda(G) = m \end{math},
\begin{math}\delta(G) = q \end{math}.
\begin{itemize}
  \item Let's assume that \begin{math} n > q \end{math}. But if some vertex has q as a minimum degree, we can just remove q vertices to disconnect graph. So, \begin{math}\varkappa(G)\end{math} equals at least q. We came to contradiction. Consequently, \begin{math} n \leq q \end{math}.
  \item Let's assume that \begin{math} m > q \end{math}. But if some vertex has at least q edges, we can just remove q edges to disconnect graph. So, \begin{math}\lambda(G)\end{math} equals at least q. We came to contradiction. Consequently, \begin{math} m \leq q \end{math}
\end{itemize}
In conclusion, \begin{math} \varkappa(G) \leq \lambda(G) \leq \delta(G)\end{math}

\section{Chartrand}



\end{document}
